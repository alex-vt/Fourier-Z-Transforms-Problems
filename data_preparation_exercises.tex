\documentclass[14pt]{article}
\usepackage[top=1cm, bottom=2cm, left=1cm, right=1cm]{geometry}
\usepackage{amsmath}
\usepackage{hyperref}
\usepackage{color}

\begin{document}
\Large

\section{The continuous signal spectrum}

\paragraph{}
The Fourier transform of a signal $ s(t) $:
\begin{equation} \label{Fourier}
	S(\omega) 
	=
	\int_{-\infty}^\infty s(t) e^{-j \omega t} {d}t
\end{equation}

\paragraph{}
The shift property of the Fourier transform (signal $ s(t + \tau) $):
\begin{equation} \label{Fourier shift}
	S(\omega) 
	=
	\int_{-\infty}^\infty s(t + \tau) e^{-j \omega t} {d}t 
	=
	e^{j \omega \tau} \int_{-\infty}^\infty s(t) e^{-j \omega t} {d}t
\end{equation}

\paragraph{}
The convolution property of the Fourier transform 
	(signal $ s(t) \circ r(t) $):
\begin{equation} \label{Fourier convolution}
	S(\omega)
	=
	\int_{-\infty}^\infty s(t) \circ r(t) e^{-j \omega t} {d}t 
	=
	\int_{-\infty}^\infty s(t) e^{-j \omega t} {d}t 
		\int_{-\infty}^\infty r(t) e^{-j \omega t} {d}t
\end{equation}

\paragraph{}
The convolution $ s(t) \circ r(t) $ is:
\begin{equation} \label{Convolution}
	s(t) \circ r(t) 
	=
	\int_{-\infty}^\infty s(\tau) r(t - \tau) {d}\tau
\end{equation}

\subsection{Exercise 1. Calculate the spectrum of the rectangular 
	signal $ s(t) $:}
\begin{equation} \label{Exercise 1.1}
	s(t) 
	=
	\left\{  
	\begin{array}{rcl}  
		A, t \in [-\frac{T}{2}, \frac{T}{2}]\\  
		0, t \notin [-\frac{T}{2}, \frac{T}{2}]\\  
	\end{array}
	\right.  
\end{equation} 

\paragraph{}
Using the Fourier transform \eqref{Fourier}:
\begin{multline} \label{Exercise 1.1 spectrum}
	S(\omega) 
	=
	\int_{-\frac{T}{2}}^{\frac{T}{2}} A e^{-j \omega t} {d}t
	=
	|t' = -j \omega t, {d}t' = -j \omega {d}t|
	=\\=
	A \int_{-\frac{j \omega T}{2}}^{\frac{j \omega T}{2}} 
		e^\frac{t'}{-j \omega} {d}t'
	=
	-\frac{A}{j \omega} (e^{-\frac{j \omega T}{2}} 
		- e^\frac{j \omega T}{2})
	=\\=
	-\frac{A}{j \omega} (cos(-\frac{\omega T}{2}) 
		+ j sin(-\frac{\omega T}{2}) - cos(\frac{\omega T}{2}) 
		- j sin(\frac{\omega T}{2}))
	=\\=
	|+cos -cos = 0|
	=
	\frac{2 A}{j \omega} j sin(\frac{\omega T}{2}) 
	=
	A T \frac{sin(\frac{\omega T}{2})}{\frac{\omega T}{2}}
\end{multline}

\subsection{Exercise 2. Calculate the spectrum of the rectangular
	signal $ s(t) $:}
\begin{equation}
	s(t) 
	=
	\left\{  
	\begin{array}{rcl}  
		A, t \in [0, T]\\  
		0, t \notin [0, T]\\  
	\end{array}
	\right.  
\end{equation} 

\paragraph{}
This signal is \eqref{Exercise 1.1}, time-shifted by 
	$ -\frac{T}{2} $:
\begin{equation}
	S(\omega) 
	\underset{\eqref{Fourier shift}}{=}
	A T e^{-j \omega \frac{T}{2}} \frac{sin(\frac{\omega T}{2})} 
		{\frac{\omega T}{2}}
\end{equation}

\begin{equation}
	|S(\omega)| 
	=
	A T \frac{sin(\frac{\omega T}{2})}{\frac{\omega T}{2}}
\end{equation}

\paragraph{}
The spectrum ripples zeros are $ \omega_k = \frac{2 k \pi}{T} $.

\subsection{Exercise 3. Calculate the spectrum of the triangular
	signal $ s(t) $:}
\begin{equation}
	s(t)
	=
	\left\{  
	\begin{array}{rcl}  
		A + \frac{2 A}{T} t, -\frac{T}{2} \leq t < 0\\  
		A - \frac{2 A}{T} t, 0 \leq t < \frac{T}{2}\\  
		0, t < -\frac{T}{2}\\   
		0, t \geq \frac{T}{2}\\  
	\end{array}
	\right.
\end{equation}

\paragraph{}
This signal is the convolution of 2 rectangular signals:
\begin{equation}
	s(t) 
	=
	\left\{  
	\begin{array}{rcl}  
		\sqrt{A}, t \in [-\frac{T}{2}, \frac{T}{2}]\\  
		0, t \notin [-\frac{T}{2}, \frac{T}{2}]\\  
	\end{array}
	\right.  
\end{equation}
which spectrums are, similarly to \eqref{Exercise 1.1 spectrum}:
\begin{equation}
	S(\omega)
	=
	\sqrt{A} \frac{sin(\frac{\omega T}{2})} {\frac{\omega T}{2}} T
\end{equation}

\paragraph{}
Using the convolution property of the Fourier transform:
\begin{equation}
	S(\omega)
	=
	\sqrt{A} \frac{sin(\frac{\omega T}{2})} {\frac{\omega T}{2}} 
	\sqrt{A} \frac{sin(\frac{\omega T}{2})} {\frac{\omega T}{2}} T 
	=
	A T (\frac{sin(\frac{\omega T}{2})}{\frac{\omega T}{2}})^2
\end{equation}

\subsection{Exercise 4. Calculate the filling factor of the
	periodical rectangular impulse signal to get the maximum power.}

\paragraph{}
The power reaches the maximum when the 1st harmonic of the signal 
	Fourier series is maximal.

Assume the period $ T $, then the filling factor is 
	$ \frac{\tau}{T} $.

\paragraph{}
The $ i $-th coefficient of the Fourier series:
\begin{equation}
	a(i)
	=
	\frac{1}{T} \int_{-\frac{T}{2}}^{\frac{T}{2}} s(t) 
		cos(\frac{2 \pi}{T} i t) {d}t
\end{equation}

\paragraph{}
The 1st coefficient of the Fourier series for the investigated
	signal:
\begin{multline}
	a(1)
	=
	\frac{1}{T} \int_{-\frac{\tau}{2}}^{\frac{\tau}{2}} A
		cos(\frac{2 \pi}{T} t) {d}t
	=
	\frac{2 A}{T} \int_{0}^{\frac{\tau}{2}} 
		cos(\frac{2 \pi}{T} t) {d}t
	=\\=
	\frac{2 A}{T \frac{2 \pi}{T}} \int_0^{\frac{\tau}{2} 
		\frac{2 \pi}{T}} cos(\frac{2 \pi}{T} t) 
		{d}(t \frac{2 \pi}{T}) 
	=\\=
	\frac{A}{\pi} sin(\frac{\tau}{2} \frac{2 \pi}{T})
	=
	\frac{A}{\pi} sin(\frac{\tau}{T} \pi) 
	-> max, at [\frac{\tau}{T} = 0.5]
\end{multline}

\subsection{Exercise 5. Calculate the spectrum of the Gaussian
	signal $ s(t) $:}
\begin{equation}
	s(t)
	=
	A e^{-\frac{t^2}{2 a^2}} 
\end{equation}

\paragraph{}
Hint: 
\begin{equation} \label{Root Pi}
	\int_{-\infty}^{\infty} e^{-x^2} {d}x
	=
	\sqrt{\pi}
\end{equation}

\paragraph{}
Then:
\begin{multline}
	S(\omega)
	=
	\int_{-\infty}^{\infty} A e^{-\frac{t^2}{2 a^2}} 
		e^{-j\omega t} {d}t 
	=
	A \int_{-\infty}^{\infty} e^{-\frac{t^2}{2 a^2} -j\omega t} {d}t 
	=\\=
	|x = \frac{t}{\sqrt{2}a}+j\frac{\omega a}{\sqrt{2}};
	x^2 = \frac{t^2}{2 a^2} + j \omega t - \frac{\omega^2 a^2}{2};
	-\frac{t^2}{2 a^2} -j\omega t = -x^2 - \frac{\omega^2 a^2}{2}|
	=\\=
	A \int_{-\infty}^{\infty} e^{-x^2 - \frac{\omega^2 a^2}{2}}
		{d}(\sqrt{2} a x) 
	=
	A \sqrt{2} a e^{-\frac{a^2 \omega^2}{2}} 		
		\int_{-\infty}^{\infty}  e^{-x^2} {d}x
	\underset{\eqref{Root Pi}}{=}
	A \sqrt{2 \pi} a e^{-\frac{a^2 \omega^2}{2}}
\end{multline}

\section{The discrete signal spectrum}

\paragraph{}
A discrete signal representation:
\begin{equation}
	s(t)
	=
	\sum_{n = -\infty}^{\infty} s(n T_s) \delta(t - n T_s)
\end{equation}
where $ T_s $ is the discretization period.

\paragraph{}
The Fourier transform of the $ \delta $-function:
\begin{equation} \label{Delta}
	S(\omega)
	=
	\int_{-\infty}^\infty \delta(t) e^{-j \omega t} {d}t
	=
	1
\end{equation}

\paragraph{}
Assume
\begin{equation} \label{Discrete}
	s[n]
	=
	s(n T_s)
\end{equation}

\paragraph{}
The Fourier transform of a discrete signal:
\begin{multline}
	S(\omega) 
	=
	\int_{-\infty}^{\infty} \sum_{n = -\infty}^{\infty} s(n T_s)
		\delta (t - n T_s) e^{-j \omega t} {d}t 
	=\\=
	\sum_{n = -\infty}^{\infty} s(n T_s) \int_{-\infty}^{\infty} 
		\delta (t - n T_s) e^{-j \omega t} {d}t
	\underset{\eqref{Fourier shift}}{=}
	\sum_{n = -\infty}^{\infty} s(n T_s) e^{-j \omega n T_s} 
		\int_{-\infty}^{\infty} \delta (t) e^{-j \omega t} {d}t 
	\underset{\eqref{Delta}}{=}\\=
	\sum_{n = -\infty}^{\infty} s(n T_s) e^{-j \omega n T_s}
	\underset{\eqref{Discrete}}{=}
	\sum_{n = -\infty}^{\infty} s[n] e^{-j \omega n}
\end{multline}

\paragraph{}
The discrete-time Fourier transform is obtained:
\begin{equation}
	S(\omega) 
	=
	\sum_{n = -\infty}^{\infty} s[n] e^{-j \omega n}
\end{equation}

\subsection{Exercise 1. Calculate the spectrum of the signal 
	$ s[n] $:}
\begin{equation}
	s[n]
	=
	\delta[n]
\end{equation}

\paragraph{}
$ \delta[n] $ is non-zero only in 0, as well as $ \delta(t) $. Hence
\begin{equation} \label{Delta DTFT}
	S(\omega) 
	=
	\sum_{n = -\infty}^{\infty} \delta[n] e^{-j \omega n} 
	= 
	\int_{-\infty}^\infty \delta(t) e^{-j \omega t} {d}t 
	=
	1
\end{equation}

\subsection{Exercise 2. Calculate the spectrum of the signal 
	$ s[n] $:}
\begin{equation}
	s[n] 
	=
	\mu[n] a^n
\end{equation}
where
\begin{equation} \label{mu}
	\mu[n]
	=
	\left\{  
	\begin{array}{rcl}  
		1, n \geq 0\\  
		0, n < 0\\  
	\end{array}
	\right.
\end{equation}

\paragraph{}
$ s[n] $ is the right-sided sequence. Hence the geometric
	progression sum formula can be used:
\begin{equation} \label{Progression}
	\sum_{n = 0}^\infty q^n 
	=
	\frac{1}{1 - q}
\end{equation}

\paragraph{}
Then,
\begin{equation}
	S(\omega) 
	=
	\sum_{n = 0}^\infty a^n e^{-j \omega n}
	=
	\sum_{n = 0}^\infty (a e^{-j \omega})^n 
	\underset{\eqref{Progression}}{=}
	\frac{1}{1 - a e^{-j \omega}}
\end{equation}

\subsection{Exercise 3. Calculate the spectrum of the signal 
	$ s[n] $:}
\begin{equation}
	s[n]
	=
	\left\{  
	\begin{array}{rcl}  
		A, 0 \leq n < M\\  
		0, n < 0\\    
		0, n \geq M\\  
	\end{array}
	\right.
\end{equation}

\begin{multline}
	S(\omega)
	=
	\sum_{n = 0}^{M - 1} A e^{-j \omega n} 
	=
	A \sum_{n = 0}^{\infty} e^{-j \omega n} 
		- A \sum_{n = M}^{\infty} e^{-j \omega n}
	\underset{\eqref{Progression}}{=}\\=
	\frac{A}{1 - e^{-j \omega}} 
		- A \sum_{n = M}^{\infty} e^{-j \omega n}
	=
	|k = n - M; n = k + M|
	=
	\frac{A}{1 - e^{-j \omega}} 
		- A \sum_{k = 0}^{\infty} e^{-j \omega (k + M)}
	=\\=
	\frac{A}{1 - e^{-j \omega}} 
		- A e^{-j \omega M} \sum_{k = 0}^{\infty} e^{-j \omega k}
	\underset{\eqref{Progression}}{=}
	\frac{A}{1 - e^{-j \omega}} 
		- \frac{A e^{-j \omega M}}{1 - e^{-j \omega}}
	=
	A \frac{1 - e^{-j \omega M}}{1 - e^{-j \omega}}
\end{multline}

\section{The Z-transform}

\paragraph{}
The Z-transform of a signal $ s[n] $:
\begin{equation} \label{Z}
	S(z)
	=
	\sum_{n = -\infty}^{\infty} s[n] z^{-n}
\end{equation}

\paragraph{}
Its region of convergence is the set of points $ z_i $ where 
	$ |S(z)| < \infty $

\paragraph{}
The inverse Z-transform of a particular type of functions $ S(z) $ 
	by partial fractions expansion method:
\begin{equation} \label{PFE-1}
	S(z)
	=
	A + \sum_{k = 1}^{n} \frac{B_k z}{z - C_k}
\end{equation}

\begin{equation} \label{PFE-2}
	s[n]
	=
	A \delta[n] + \mu[n] \sum_{k = 1}^{n} B_k (C_k)^n, n \geq 0
\end{equation}
where $ \mu[k] $ is \eqref{mu}.

\subsection{Exercise 1. Calculate the Z-transform of the signal 
	$ s[n] $:}
\begin{equation}
	s[n]
	=
	\delta[n]
\end{equation}

\begin{equation}
	S(z) 
	\underset{\eqref{Z}}{=}
	\sum_{n = -\infty}^{\infty} \delta[n] z^{-n}
	=
	|z = e^{-j \omega}|
	=
	\sum_{n = -\infty}^{\infty} \delta[n] e^{-j \omega n} 
	\underset{\eqref{Delta DTFT}}{=}
	\int_{-\infty}^\infty \delta(t) e^{-j \omega t} {d}t 
	=
	1
\end{equation}

\paragraph{}
As the obtained Z-transform is constant, its region of convergence 
	is the whole complex plane. 

\subsection{Exercise 2. Calculate the Z-transform of the signal 
	$ s[n] $:}
\begin{equation}
	s[n] 
	=
	\mu[n]
\end{equation}
where $ \mu[n] $ is \eqref{mu}.

\paragraph{}
$ \mu[n] $ causes $ S(z) $ to be the right-sided sequence:
\begin{equation}
	S(z)
	=
	\sum_{n = 0}^{\infty} z^{-n} 
	\underset{\eqref{Progression}}{=}
	\frac{1}{1 - z^{-1}}
\end{equation}

\paragraph{}
Region of convergence: $ |z| > 1 $, as defined by $ q $ restrictions 
	in \eqref{Progression} for complex numbers.

\subsection{Exercise 3. Calculate the Z-transform of the signal 
	$ s[n] $:}
\begin{equation}
	s[n]
	=
	\mu[n] a^n
\end{equation}

\paragraph{}
The right-sided sequence:
\begin{equation}
	S(z) 
	=
	\sum_{n = 0}^\infty a^n z^{-n}
	=
	\sum_{n = 0}^\infty (a z^{-1})^n 
	\underset{\eqref{Progression}}{=}
	\frac{1}{1-a z^{-1}}
\end{equation}

\paragraph{}
Region of convergence: $ |z| > |a| $.

\subsection{Exercise 4. Calculate the Z-transform of the signal 
	$ s[n] $:}
\begin{equation}
	s[n]
	=
	-\mu[-n - 1] a^n
\end{equation}

\paragraph{}
$ \mu[-n - 1] = 1 $ when $ n \leq -1 $.

\paragraph{}
Then,
\begin{multline}
	S(z)
	=
	\sum_{n = -\infty}^{-1} -a^n z^{-n}
	=
	|k = -n|
	=
	\sum_{k = 1}^{\infty} -a^{-k} z^{k}
	=
	-\sum_{k = 1}^{\infty} (a^{-1} z)^{k}
	=\\=
	-a^{-1} z \frac{1}{1 - a^{-1} z}
	=
	-a^{-1} z \frac{a z^{-1}}{a z^{-1} - 1}
	=
	\frac{a^{-1} z a z^{-1}}{1 - a z^{-1}}
	=
	\frac{1}{1 - a z^{-1}}
\end{multline}

\paragraph{}
The sequence is left-sided. Its region of convergence: $ |z| < |a| $.

\subsection{Exercise 5. Calculate the Z-transform of the signal 
	$ s[n] $:}
\begin{equation}
	s[n]
	=
	\mu[n] cos(\omega n)
\end{equation}

\paragraph{}
Euler's formula for $ cos (x) $:
\begin{equation}  \label{Cos Euler}
	cos(x)
	=
	Re[e^{i x}]
	=
	\frac{e^{i x} - e^{-i x}}{2}
\end{equation}

\paragraph{}
Hence
\begin{equation}
	s[n]
	=
	\mu[n] \frac{e^{i \omega n}}{2} 
		+ \mu[n] \frac{e^{-i \omega n}}{2}
\end{equation}

\paragraph{}
The Z-transform:
\begin{multline}
	S(z)
	=
	\frac{1}{2} \sum_0^\infty e^{i \omega n} z^{-n} 
		+ \frac{1}{2} \sum_0^\infty e^{-i \omega n} z^{-n} 
	=
	\frac{1}{2} \sum_0^\infty (e^{i \omega} z^{-1})^n 
		+ \frac{1}{2} \sum_0^\infty (e^{-i \omega} z^{-1})^n 
	\underset{\eqref{Progression}}{=}\\=
	\frac{\frac{1}{2}}{1 - e^{i \omega} z^{-1}} 
		+ \frac{\frac{1}{2}}{1 - e^{-i \omega} z^{-1}}
	=
	\frac{\frac{1}{2} (1 - e^{-i \omega} z^{-1})
		+ \frac{1}{2} (1 - e^{i \omega} z^{-1})}{1 - e^{i \omega}
		z^{-1} - e^{-i \omega} z^{-1} + e^{i \omega} z^{-1} 
		e^{-i \omega} z^{-1}}
	=\\=
	\frac{\frac{1}{2} + \frac{1}{2} - \frac{1}{2}(e^{-i \omega} 
	+ e^{i \omega}) z^{-1}}{1 - 2 \frac{1}{2}(e^{-i \omega} 
	+ e^{i \omega}) z^{-1} + z^{-2}}
	\underset{\eqref{Cos Euler}}{=}
	\frac{1 - cos(\omega) z^{-1}}{1 - 2 cos(\omega) z^{-1} + z^{-2}}
\end{multline}

\paragraph{}
The sequence is right-sided. Region of convergence: $ |z| > 1 $.

\subsection{Exercise 6. Calculate the Z-transform of the signal 
	$ s[n] $:}
\begin{equation}
	s[n]
	=
	\left\{  
	\begin{array}{rcl}  
		A, 0 \leq n < M\\  
		0, n < 0\\    
		0, n \geq M\\  
	\end{array}
	\right.
\end{equation}

\begin{multline}
	S(z)
	=
	\sum_{n = 0}^{M - 1} A z^{-n} 
	=
	A \sum_{n = 0}^{\infty} z^{-n} - A \sum_{n = M}^{\infty} z^{-n}
	\underset{\eqref{Progression}}{=}\\=
	\frac{A}{1 - z^{-1}} - A \sum_{n = M}^{\infty} z^{-n}
	=
	|k = n-M;n = k+M|
	=
	\frac{A}{1 - z^{-1}} - A \sum_{k = 0}^{\infty} z^{-(k + M)}
	=\\=
	\frac{A}{1 - z^{-1}} - A z^{-M} \sum_{k = 0}^{\infty} z^{-k}
	\underset{\eqref{Progression}}{=}
	\frac{A}{1 - z^{-1}} - \frac{A z^{-M}}{1 - z^{-1}}
	=
	A \frac{1 - z^{-M}}{1 - z^{-1}}
\end{multline}

\paragraph{}
As the sequence is limited, its region of convergence is: 
	$ |z| > 0 $.

\subsection{Exercise 7. Calculate (by partial fraction expansion
	method) the Z-transform of $ S(z) $:}
\begin{equation}
	S(z)
	=
	\frac{1}{1 - 0.3 z^{-1} - 0.1 z^{-2}}
\end{equation}

\paragraph{}
Decomposing S(z) to form \eqref{PFE-1}:
\begin{multline} \label{PQ}
	S(z)
	=
	\frac{1}{1 - 0.3 z^{-1} - 0.1 z^{-2}}
	=
	\frac{1}{(1 - 0.5 z^{-1})(1 + 0.2 z^{-1})}
	=\\=
	\frac{P}{1 - 0.5 z^{-1}} + \frac{Q}{1 + 0.2 z^{-1}}
	=
	\frac{P(1 + 0.2 z^{-1}) + Q(1 - 0.5 z^{-1})}{(1 - 0.5 z^{-1})
		(1 + 0.2 z^{-1})}
\end{multline}

\paragraph{}
Finding the parameters $ P $, $ Q $ of the numerator of \eqref{PQ}:
\begin{equation}
	P(1 + 0.2 z^{-1}) + Q(1 - 0.5 z^{-1})
	=
	1
\end{equation}

\begin{equation}
	\left\{  
	\begin{array}{rcl}  
		P + Q = 1\\  
		0.2 P z^{-1} -0.5 Q z^{-1} = 0\\    
	\end{array}
	\right.
\end{equation}

\begin{equation}
	\left\{  
	\begin{array}{rcl}  
		P + Q = 1\\  
		0.2 P - 0.5 Q = 0\\    
	\end{array}
	\right.
\end{equation}

\begin{equation}
	\left\{  
	\begin{array}{rcl}  
		Q = 1 - P\\  
		0.2 P  -0.5 (1-P) = 0\\    
	\end{array}
	\right.
\end{equation}

\begin{equation}
	\left\{  
	\begin{array}{rcl}  
		Q = 1 - P\\  
		0.2 P - 0.5 + 0.5 P = 0\\    
	\end{array}
	\right.
\end{equation}

\begin{equation}
	\left\{  
	\begin{array}{rcl}  
		Q = 1 - P\\  
		0.7 P = 0.5\\    
	\end{array}
	\right.
\end{equation}

\begin{equation}
	\left\{  
	\begin{array}{rcl}  
		Q = \frac{2}{7}\\  
		P = \frac{5}{7}\\    
	\end{array}
	\right.
\end{equation}

\paragraph{}
Continuing \eqref{PQ}:
\begin{equation}
	S(z)
	=
	\frac{\frac{5}{7}}{1 - 0.5 z^{-1}} 
		+ \frac{\frac{2}{7}}{1 + 0.2 z^{-1}}
	=
	\frac{\frac{5}{7}}{1 - \frac{0.5}{z}} 
		+ \frac{\frac{2}{7}}{1 + \frac{0.2}{z}}
	=
	\frac{\frac{5}{7}z}{z - 0.5} + \frac{\frac{2}{7}z}{z + 0.2}
\end{equation}

\paragraph{}
Hereby, according to \eqref{PFE-2}:
\begin{equation}
	s[n]
	=
	\frac{5}{7} 0.5^n + \frac{2}{7} (-0.2)^n, n \geq 0
\end{equation}

\end{document}





